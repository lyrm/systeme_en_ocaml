\documentclass{beamer}
\usetheme{metropolis}           % Use metropolis them
\usecolortheme{metropolis}
\setbeamersize{text margin left=10mm,text margin right=5mm}
\usepackage[utf8]{inputenc}
\usepackage[french]{babel}
\usepackage[T1]{fontenc}
\usepackage{lmodern}
\usepackage{enumitem}
\usepackage{amsmath,amssymb,amsfonts}
\usepackage{xcolor}
\usepackage{colortbl}
\usepackage{tikz}
\usepackage{array}
\usepackage{listings}
% Configuring listings for OCaml.

% Comments in blue.
\newcommand{\ocamlcommentstyle}{\color{blue}}

\lstdefinelanguage{ocaml}[Objective]{Caml}{
  % Fix errors in the default definition of ocaml.
  deletekeywords={closed,ref},
  morekeywords={initializer,define,CONCAT}, % really #define
  % General settings.
  flexiblecolumns=false,
  showstringspaces=false,
  framesep=5pt,
  commentstyle=\ocamlcommentstyle,
  % By default, we use a small font.
  basicstyle=\tt\small,
  numberstyle=\footnotesize,
  % LaTeX escape.
  escapeinside={$}{$},
}

% An abbreviation for \lstinline, with a normal font size.
% To be used in the text of the paper.
\def\oc{\lstinline[language=ocaml,basicstyle=\tt,flexiblecolumns=true]}

% With quotes.
\def\qoc|#1|{``\oc|#1|''}

\lstset{language=ocaml}
\usepackage{mdframed}
\usepackage{centernot}
\usepackage{pifont}

\patchcmd{\thebibliography}{\section*{\refname}}{}{}{}

\setbeamertemplate{blocks}[rounded][shadow=true]
\setbeamertemplate{itemize subitem}[bullet]
\newcommand{\added}[1]{\textcolor{gray!60!cyan}{#1}}
\newcommand{\subtt}[1]{\textcolor{gray!30!cyan}{#1}}

\newcommand\tgray[2]{\textcolor<#1>{gray!30}{#2}}
\newcommand\torange[2]{\textcolor<#1>{orange}{#2}}
\newcommand\tcyan[2]{\textcolor<#1>{cyan}{#2}}
\newcommand\tgreen[2]{\textcolor<#1>{cyan!10!green}{#2}}
\newcommand\tred[2]{\textcolor<#1>{red}{#2}}

\title{Programmation système en Ocaml : introduction et cas d'usage.}
%\date{\today}
\author{Carine Morel et Lucas Pluvinage, Tarides}

\begin{document}
\maketitle

% ---------------------------------------------------------------- %
%\section{Contexte et problématiques}

\begin{frame}{Contexte}
  \begin{itemize}
  \item<1-> Inférence de types :
    \begin{itemize}[label=$-$]
    \item classiquement : algorithmes $\mathcal{J}$ ou $\mathcal{W}$ de Milner (1978)
    \item ici : inférence de types par contrainte ( F. Pottier et D. Rémy, 2005)
    \end{itemize}
    %
  \item<2->
    %
  \item<2-> \subtt{Différences ?} \\
    Deux phases :  génération et résolution de contraintes.
  \item<2-> \subtt{Intérêts ?} \\Modularité de l'algorithme

  \item<3->
  \item<3-> \textbf{Haskell} : inférence de types par contrainte
  \item<3-> \textbf{OCaml} : version très étendue de l'algorithme $\mathcal{W}$
  \end{itemize}

\end{frame}

\begin{frame}{Bonjour}
    

    
\end{frame}
\end{document}
