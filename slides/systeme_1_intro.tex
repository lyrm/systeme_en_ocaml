
\begin{frame}{Introduction : programmation système en OCaml}

    \begin{itemize}[label=\small\ding{114}]
        \item Des modules bas niveau (cf \url{ocaml.org/api})
            \begin{itemize}[label=\small\ding{118}]
                \item \texttt{Sys} : fonctions communes à Unix et aux autres OS sous lesquels tourne OCaml,
                \item \texttt{Unix} : fonctions spécifiques à Unix.
            \end{itemize}
        \item Des sur-couches sur le module \texttt{Unix} :
            \begin{itemize}[label=\small\ding{118}]
                \item certaines fonctions de la \texttt{Sdtlib}, 
                \item \texttt{Lwt.Unix}
            \end{itemize}
        \item Des modules utilitaires comme \texttt{Filename}
    \end{itemize}

\end{frame}


\begin{frame}{Introduction : le mini-shell}

    \subtt{Objectif :} programmer un shell en utilisant le module \texttt{Unix} 
    
    Les fonctionnalités et commandes que l'on va implémenter :
    \begin{itemize}[label=\small\ding{114}]
        \item<1-> Manipulation des fichiers et répertoires : \texttt{ls}, \texttt{mkdir}, \texttt{ln}, \texttt{cat}
        \item<1-> Modification du répertoire courant : \texttt{cd}
        \item<1-> Re-directions de flux :  $>$, $<$
        \item<1-> Tube : $|$

        %\item<2-> Pour aller plus loin : 
        %    \begin{itemize}[label=$-$]
        %        \item un peu de \textit{réseau}
        %        \item les sur-couches au module \texttt{Unix} : \textit{Lwt / Bos}
        %    \end{itemize}
    \end{itemize}
\end{frame}
